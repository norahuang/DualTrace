\startchapter{Background}
\label{chapter:Bac}
In this section, I summarize the background knowledge or information that related to this work. First I generally describe software vulnerability. Second, I discuss the general definition of communication among programs and their categorization. Third, I introduce some tools for assembly level program debugging and analysis. Finally I introduce Atlantis, the existing assembly level execution trace analysis environment, on which the implementation of this work based.

\section{Software Vulnerability}
Software vulnerability detection is one of the use cases of the communication analysis with assembly level execution traces. The understanding of fundamentals of software vulnerability is necessary to comprehend some implicit concepts or design intention through out this thesis. 

Vulnerabilities, from the point of view of software security, are specific flaws or oversights in a program that can be exploited by attackers to do something malicious expose such as modify sensitive information, disrupt or destroy a system, or take control of a computer system or program. They are considered to be a subset of bugs. Input and Data Flow, interface and exceptional condition handling where vulnerabilities are most likely to surface in software and memory corruption is one of the most common vulnerabilities. The awareness of these two facts would make the security auditing and vulnerabilities detection have more clear focus. \cite{dowd_art_2006}

\section{Program Communications}
Programs can communicate with each other via diverse mechanisms. The communication happens among processes is known as inter-process communication. This refers to the mechanisms an operating system provides the process to share data with each other. It includes methods such as signal, socket, message queue, shared memory and so on.\cite{garrido2000inter} This communications can happen over network or inside a device. Based on their reliability, the communication methods can be simply divided into two categories: reliable communication and unreliable communication. In this work, communication methods belong to all both categories has been covered. However, I only discuss the message based communication methods while leave the control based communication, like remote function call for the future.

\section{Program Execution Tracing in Assembly Level}
The communication analysis discuss throughout this thesis is based on the assembly traces. Thurs capturing of the execution traces became a prerequisite of this work. DRDC has its own home-made tracer, the traces from which are used in the experiments of this research. However, the model and algorithms developed in this research is not limited with this specific home-made tracer. Any tracer that can capture sufficient information according to the model can serve this purpose.

There are many tools that can trace a running program in assembly instruction level.  IDA pro \cite{eagle_ida_2008} is a widely used tool in reverse engineering which can capture and analysis system level execution trace. Giving open plugin APIs, IDA pro allows plugin such as Codemap \cite{_c0demap/codemap:_????} to provide more sufficient features for "run-trace" visualization. PIN\cite{_pin_????} as a tool for instrumentation of programs, provides a rich API which allows users to implement their own tool for instruction trace and memory reference trace. Other tools like Dynamic \cite{brueningqz} and OllyDbg\cite{yuschuk2007ollydbg} also provide the debugging and tracing functionality in assembly level. 

\section{Atlantis}
Atlantis is a trace analysis environment developed in Chisel. It can support analysis for multi-gigabyte assembly traces. There are several features distinct it from all other existing tools and make it particularly successful in large scale trace analysis. These features are 1) reconstruction and navigation the memory state of a program at any point in a trace; b) reconstruction and navigation of system functions and processes; and c) a powerful search facility to query and navigate traces. The work of this thesis is not a extension of Atlantis. But it take advantages of Atlantis by reusing it existing functionality to assist the dual\_trace analysis.




