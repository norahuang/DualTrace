\startfirstchapter{Introduction}
\label{chapter:introduction}

Many network application vulnerabilities occur not just in one application, but in how they interact with other systems. These kinds of vulnerabilities can be difficult to analyze. Dual-trace analysis is one approach that helps the security engineers to detect the vulnerabilities in the interactive software. A dual-trace consist of two execution traces that are generated from two interacting applications. Each of these traces contains information including CPU instructions, register and memory changes of the running application. Communication information of the interacting applications is captured as the register or memory changes on their respective traced sides. 

This work is focusing on helping reverse engineers for interacting software vulnerabilities detection. We first investigated and modeled four types of commonly used channels in Windows communication foundation in order to help the reverse engineers to understand the APIs, the scenarios and the assembly trace related perspectives of these channels. Then we built a tool prototype for the communication event locating and visualization of dual-traces. Finally, we design an experiment to test our prototype and evaluate its practicality. 

{add an section to summarize the conclusion later}
