\startfirstchapter{Introduction}
\label{chapter:introduction}
This thesis present a method for communication analysis out of the captured assembly level execution traces. The dual\_trace being discussed thought out this thesis consist of two assembly level execution traces of two running program. This work aims to help reverse engineers to understand the behavior of the interacting programs in the assembly level. There are many reasons for investigating programs in assembly level. Lacking of source code and trying to understanding the memory usage are one among those reasons.

Many network application vulnerabilities occur while one program is running isolatedly, but while it is interacting with other systems. These vulnerabilities can be difficult to analyze and detect. Dual\_trace analysis is an approach that helps the security engineers to detect the vulnerabilities in the interactive programs. The two traces in dual\_trace contain information including CPU instructions, register and memory changes, system calls, modules and threads of the running application. Communication information of the interacting programs can be retrieved from the captured information in the dual\_trace. 

In this work, I first modeled the communication between two programs. This model defines the communication objects under investigate and identified in the dual\_trace. In simplification, a communication is completed between two programs each of which corresponds to a sequence of events. An event can be one four different event type(channel open, data send, data receive and channel close). However the actual events are various according to the communication method and the user concern. Then, I modeled the dual\_trace, excluding the irrelevant information while abstracting the information related to the communications. Furthermore, I investigated the implementation of the communication methods in Windows. The result of the investigation provides some guidelines and examples of how to draw the concerned event list as described in the communication model which is the prerequisite of the communication identifications. By matching the elements in the communication model and the dual\_trace model, I developed the communication identification algorithms. To provide more concrete idea, I implemented the communication identification features in Atlantis\cite{huang2017atlantis}, an assembly execution trace analysis environment.

Finally, I tested the models and the algorithms by two experiments. The experiment result shows that 1) the existing dual\_trace contain sufficient information for communication analysis based on the designed models. In addition. 2) the developed algorithm can effectively identify the communications from the dual\_trace. This work is a collaborative work with our research partner DRDC. In the experiment tests, DRDC captured the dual\_traces with its home-make pin tool of the programs running in DRDC's environment. I conducted the analysis of the provided dual\_traces as long as the dynamic libraries of the running environment locally.
