\startappendix{Additional Information}
\label{chapter:appendix}

\section{Terminology}\label{term}
\textbf{Endpoint:}\\
An instance in a program at which a stream of data are sent or received (or both). It usually is identified by the handle of a specific communication method in the program. Such as a socket handle of TCP or UDP or a file handle of the named piped channel.\\
\textbf{Channel:}\\
A conduit connected two endpoints through  which data can be sent and received\\
\textbf{Channel open event:}\\
Operation to create and connect an endpoint to a specific channel\\
\textbf{Channel close event:}\\
Operation to disconnect and delete the endpoint from the channel.\\
\textbf{Send event:}\\
Operation to send a trunk of data from one endpoint to the other through  the channel.\\
\textbf{Receive event:}\\
Operation to receive a trunk of data at one endpoint from the other through the channel.\\
\textbf{Channel open stream:}\\
A set of all channel open events regarding to a specific endpoint.\\
\textbf{Channel close stream:}\\
A set of all channel close events regarding to a specific endpoint.\\
\textbf{Send stream:}\\
A set of all send events regarding to a specific endpoint.\\
\textbf{Receive stream:}\\
A set of all receive events regarding to a specific endpoint.\\
\textbf{Stream:}\\
A stream consist of a channel open stream, a channel close stream, a send stream and a receive stream. All of these streams regard to the same endpoint.


\section{Microsoft* x64 Calling Convention for C/C++}\label{convention}
\begin{enumerate}  
\item RCX, RDX, R8, R9 are used for integer and pointer arguments in that order left to right.
\item XMM0, 1, 2, and 3 are used for floating point arguments.
\item Additional arguments are pushed on the stack left to right. \ldots 
\item Parameters less than 64 bits long are not zero extended; the high bits contain garbage.
\item Integer return values (similar to x86) are returned in RAX if 64 bits or less.
\item Floating point return values are returned in XMM0.
\item Larger return values (structs) have space allocated on the stack by the caller, and RCX then contains a pointer to the return space when the callee is called. Register usage for integer parameters is then pushed one to the right. RAX returns this address to the caller.
\end{enumerate}

