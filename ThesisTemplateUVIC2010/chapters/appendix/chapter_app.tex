\startappendix{Additional Information}
\label{chapter:appendix}

\section{Terminology}\label{term}
\textbf{Endpoint:}\\
An instance in a program at which a stream of data are sent or received (or both). It usually is identified by the handle of a specific communication method in the program. Such as a socket handle of TCP or UDP or a file handle of the named piped channel.\\
\textbf{Channel:}\\
A conduit connected two endpoints through  which data can be sent and received\\
\textbf{Channel open event:}\\
Operation to create and connect an endpoint to a specific channel\\
\textbf{Channel close event:}\\
Operation to disconnect and delete the endpoint from the channel.\\
\textbf{Send event:}\\
Operation to send a trunk of data from one endpoint to the other through  the channel.\\
\textbf{Receive event:}\\
Operation to receive a trunk of data at one endpoint from the other through the channel.\\
\textbf{Channel open stream:}\\
A set of all channel open events regarding to a specific endpoint.\\
\textbf{Channel close stream:}\\
A set of all channel close events regarding to a specific endpoint.\\
\textbf{Send stream:}\\
A set of all send events regarding to a specific endpoint.\\
\textbf{Receive stream:}\\
A set of all receive events regarding to a specific endpoint.\\
\textbf{Stream:}\\
A stream consist of a channel open stream, a channel close stream, a send stream and a receive stream. All of these streams regard to the same endpoint.

