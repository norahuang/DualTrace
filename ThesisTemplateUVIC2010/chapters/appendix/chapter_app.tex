\startappendix{Additional Information}
\label{chapter:appendix}

\section{Terminology}\label{term}
\textbf{Endpoint:}\\
An instance in a program at which a stream of data are sent or received (or both). It usually is identified by the handle of a specific communication method in the program. Such as a socket handle of TCP or UDP or a file handle of the named piped channel.\\
\textbf{Channel:}\\
A conduit connected two endpoints through  which data can be sent and received\\
\textbf{Channel open event:}\\
Operation to create and connect an endpoint to a specific channel\\
\textbf{Channel close event:}\\
Operation to disconnect and delete the endpoint from the channel.\\
\textbf{Send event:}\\
Operation to send a trunk of data from one endpoint to the other through  the channel.\\
\textbf{Receive event:}\\
Operation to receive a trunk of data at one endpoint from the other through the channel.\\
\textbf{Channel open stream:}\\
A set of all channel open events regarding to a specific endpoint.\\
\textbf{Channel close stream:}\\
A set of all channel close events regarding to a specific endpoint.\\
\textbf{Send stream:}\\
A set of all send events regarding to a specific endpoint.\\
\textbf{Receive stream:}\\
A set of all receive events regarding to a specific endpoint.\\
\textbf{Stream:}\\
A stream consist of a channel open stream, a channel close stream, a send stream and a receive stream. All of these streams regard to the same endpoint.


\section{Microsoft* x64 Calling Convention for C/C++}\label{convention}
\begin{enumerate}  
\item RCX, RDX, R8, R9 are used for integer and pointer arguments in that order left to right.
\item XMM0, 1, 2, and 3 are used for floating point arguments.
\item Additional arguments are pushed on the stack left to right. \ldots 
\item Parameters less than 64 bits long are not zero extended; the high bits contain garbage.
\item Integer return values (similar to x86) are returned in RAX if 64 bits or less.
\item Floating point return values are returned in XMM0.
\item Larger return values (structs) have space allocated on the stack by the caller, and RCX then contains a pointer to the return space when the callee is called. Register usage for integer parameters is then pushed one to the right. RAX returns this address to the caller.
\end{enumerate}

\section{Example of Configuration File for Communication Methods' Function Set}\label{funcset}
\begin{lstlisting}
[
   {
      "communicationMethod": "NamedPipe",
      "funcList": [
         {
            "retrunValReg": {
               "name": "RAX",
               "valueOrAddress": true
            },
            "valueInputReg": {
               "name": "RCX",
               "valueOrAddress": false
            },
            "functionName": "CreateNamedPipeA",
            "createHandle": true,
            "type": "open"
         },
         {
            "retrunValReg": {
               "name": "RAX",
               "valueOrAddress": true
            },
            "valueInputReg": {
               "name": "RCX",
               "valueOrAddress": false
            },
            "functionName": "CreateFileA",
            "createHandle": true,
            "type": "open"
         },
		 {
            "retrunValReg": {
               "name": "RAX",
               "valueOrAddress": value
            },
            "valueInputReg": {
               "name": "RCX",
               "valueOrAddress": value
            },
            "memoryInputReg": {
               "name": "RDX",
               "valueOrAddress": address
            },
            "memoryInputLenReg": {
               "name": "R8",
               "valueOrAddress": value
            },
            "functionName": "WriteFile",
            "createHandle": false,
            "type": "send"
         },
         {
            "retrunValReg": {
               "name": "RAX",
               "valueOrAddress": value
            },
            "valueInputReg": {
               "name": "RCX",
               "valueOrAddress": value
            },
            "memoryOutputReg": {
               "name": "RDX",
               "valueOrAddress": address
            },
            "memoryOutputBufLenReg": {
               "name": "R8",
               "valueOrAddress": value
            },
            "functionName": "ReadFile",
            "createHandle": false,
            "type": "recv",
            "outputDataAddressIndex": "NamedPipeChannelRDX"
         },
         {
            "retrunValReg": {
               "name": "RAX",
               "valueOrAddress": value
            },
            "valueInputReg": {
               "name": "RCX",
               "valueOrAddress": value
            },
            "memoryOutputReg": {
               "name": "RDX",
               "valueOrAddress": address
            },
            "functionName": "GetOverlappedResult",
            "createHandle": false,
            "type": "check",
            "outputDataAddressIndex": "NamedPipeChannelRDX"
         },
		 {
            "retrunValReg": {
               "name": "RAX",
               "valueOrAddress": value
            },
            "valueInputReg": {
               "name": "RCX",
               "valueOrAddress": value
            },
            "functionName": "CloseHandle",
            "createHandle": false,
            "type": "na"
         }
      ]
   }
]
\end{lstlisting}

\section{Open View with Two Parallel Editors programmatically}\label{paralleleditor}
\subsection{The Editor Area Split Handler}
\begin{lstlisting}
public class OpenDualEditorsHandler extends AbstractHandler {
	EModelService ms;
	EPartService ps;
	WorkbenchPage page;

	  
    public Object execute(ExecutionEvent event) throws ExecutionException {
		IEditorPart editorPart = HandlerUtil.getActiveEditor(event);
		if (editorPart == null) {
			Throwable throwable = new Throwable("No active editor");
			BigFileApplication.showErrorDialog("No active editor", "Please open one file first", throwable);
			return null;
		}

		MPart container = (MPart) editorPart.getSite().getService(MPart.class);
		MElementContainer m = container.getParent();
		if (m instanceof PartSashContainerImpl) {
			Throwable throwable = new Throwable("The active file is already opened in one of the parallel editors");
			BigFileApplication.showErrorDialog("TThe active file is already opened in one of the parallel editors",
					"The active file is already opened in one of the parallel editors", throwable);
			return null;
		}
		IFile file = getPathOfSelectedFile(event);

		IEditorDescriptor desc = PlatformUI.getWorkbench().getEditorRegistry().getDefaultEditor(file.getName());
		try {
			IFileUtils fileUtil = RegistryUtils.getFileUtils();
			File f = BfvFileUtils.convertFileIFile(file);
			f = fileUtil.convertFileToBlankFile(f);
			IFile convertedFile = ResourcesPlugin.getWorkspace().getRoot().getFileForLocation(Path.fromOSString(f.getAbsolutePath()));
			convertedFile.getProject().refreshLocal(IResource.DEPTH_INFINITE, null);
			if (!convertedFile.exists()) {
				createEmptyFile(convertedFile);
			}

			IEditorPart containerEditor = HandlerUtil.getActiveEditorChecked(event);
			IWorkbenchWindow window = HandlerUtil.getActiveWorkbenchWindowChecked(event);
			ms = window.getService(EModelService.class);
			ps = window.getService(EPartService.class);
			page = (WorkbenchPage) window.getActivePage();
			IEditorPart editorToInsert = page.openEditor(new FileEditorInput(convertedFile), desc.getId());
			splitEditor(0.5f, 3, editorToInsert, containerEditor, new FileEditorInput(convertedFile));
			window.getShell().layout(true, true);
			

		} catch (CoreException e) {
			e.printStackTrace();
		}

		return null;
	}

    private void createEmptyFile(IFile file) {
		byte[] emptyBytes = "".getBytes();
		InputStream source = new ByteArrayInputStream(emptyBytes);
		try {
			createParentFolders(file);
			if(!file.exists()){
				file.create(source, false, null);
			}
		} catch (CoreException e) {
			e.printStackTrace();
		}finally{
			try {
				source.close();
			} catch (IOException e) {
				// Don't care
			}
		}
	}

	private void splitEditor(float ratio, int where, IEditorPart editorToInsert, IEditorPart containerEditor,
			FileEditorInput newEditorInput) {
		MPart container = (MPart) containerEditor.getSite().getService(MPart.class);
		if (container == null) {
			return;
		}

		MPart toInsert = (MPart) editorToInsert.getSite().getService(MPart.class);
		if (toInsert == null) {
			return;
		}

		MPartStack stackContainer = getStackFor(container);
		MElementContainer<MUIElement> parent = container.getParent();
		int index = parent.getChildren().indexOf(container);
		MStackElement stackSelElement = stackContainer.getChildren().get(index);

		MPartSashContainer psc = ms.createModelElement(MPartSashContainer.class);
		psc.setHorizontal(true);
		psc.getChildren().add((MPartSashContainerElement) stackSelElement);
		psc.getChildren().add(toInsert);
		psc.setSelectedElement((MPartSashContainerElement) stackSelElement);

		MCompositePart compPart = ms.createModelElement(MCompositePart.class);
		compPart.getTags().add(EPartService.REMOVE_ON_HIDE_TAG);
		compPart.setCloseable(true);
		compPart.getChildren().add(psc);
		compPart.setSelectedElement(psc);
		compPart.setLabel("dual-trace:" + containerEditor.getTitle() + " and " + editorToInsert.getTitle());

		parent.getChildren().add(index, compPart);
		ps.activate(compPart);

	}

	private MPartStack getStackFor(MPart part) {
		MUIElement presentationElement = part.getCurSharedRef() == null ? part : part.getCurSharedRef();
		MUIElement parent = presentationElement.getParent();
		while (parent != null && !(parent instanceof MPartStack))
			parent = parent.getParent();

		return (MPartStack) parent;
	}


	private IFile getPathOfSelectedFile(ExecutionEvent event) {
		IWorkbenchWindow window = PlatformUI.getWorkbench().getActiveWorkbenchWindow();
		if (window != null) {
			window = HandlerUtil.getActiveWorkbenchWindow(event);
			IStructuredSelection selection = (IStructuredSelection) window.getSelectionService().getSelection();
			Object firstElement = selection.getFirstElement();
			if (firstElement instanceof IFile) {
				return (IFile) firstElement;
			}
			if (firstElement instanceof IFolder) {
				IFolder folder = (IFolder) firstElement;
				AtlantisBinaryFormat binaryFormat = new AtlantisBinaryFormat(
						folder.getRawLocation().makeAbsolute().toFile());
				// arbitrary, just any file in the binary set is needed
				return AtlantisFileUtils.convertFileIFile(binaryFormat.getExecVtableFile());
			}
		}
		return null;
	}
}
\end{lstlisting}

\subsection{Get the Active Parallel Editors}
\begin{lstlisting}
IEditorPart editorPart = PlatformUI.getWorkbench().getActiveWorkbenchWindow().getActivePage().getActiveEditor();
		MPart container = (MPart) editorPart.getSite().getService(MPart.class);
		MElementContainer m = container.getParent();
		if (!(m instanceof PartSashContainerImpl)) {
			Throwable throwable = new Throwable("This is not a dual-trace");
			BigFileApplication.showErrorDialog("This is not a dual-trace!", "Open a dual-trace First", throwable);
			return;
		}

		MPart editorPart1 = (MPart) m.getChildren().get(0);
		MPart editorPart2 = (MPart) m.getChildren().get(1);
\end{lstlisting}