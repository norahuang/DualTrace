\startchapter{Conclusions and Future Work}
\label{concl}
In this thesis, I present the designed communication identification models. This model consist of three sub models, communication definition model, dual\_trace model and identification matching model for matching the elements in the dual\_trace to the elements in the communication definition. This model provide the guideline for communication analysis for software security engineers and researchers in assembly execution trace level. By understanding this model, it should be possible for them to conduct their own communication analysis, identifying the concerned communication methods from the captured execution traces of interacting programs.

I also developed the essential algorithms for the communication identification. The high level algorithm is generalizable for all communication methods' identification while  the stream identification and matching algorithm are distinct for each communication method according to their channel open and data transfer mechanisms. However, the developed algorithms provides clear and referable examples to develop your own algorithm for communication methods which are not discussed in this thesis.

On top of the existing execution trace analysis environment Atlantis, I implemented the communication identification features. The design provides the users a way to extend their concerned communication methods through the configuration file. The extended user interface allows the users to conduct the communication and stream identification from the dual\_traces and navigate back from the identified result to the views of the trace in Atlantis. This feature prototype is a novel feature for conducting multiple trace analysis for reverse engineering at the time when this thesis was written. 

The experiments conducted in this work preliminary proves the usability of the model and the algorithms. It also demonstrate the limitation for eliminating the false negative error of the communication identification. Other information is needed to assist the identification in order to improve its accuracy.

This thesis illustrates the novel idea and approach for dynamic program analysis which considerate the interaction of two programs. This idea is valuable due to the fact that programs or malware in the real world work collaboratively. The analysis of the communication and interaction of the programs provide more reliable information for vulnerability detection and program analysis.

Future work can be divided in two directions. One is extending the model to be more generalize for all kinds of interaction but not only the message transferring communications while the other is conducting user studies of the model and feature design to get a more concrete result of their usefulness.

