\startchapter{Conclusions and Future Work}
\label{concl}
\section{Conclusions}
In this section, we specify the limitations of the current prototype and the reasons for them. 
\subsection{Event Status: Success or Fail}
In current prototype, we only consider the success cases. For the Fail case, since the message was not successfully sent or received, there are high chance that they are not existed in the memory of the trace. From the assembly level trace, if the message was not traced in the memory, there is no way to match the sent/received message pair in the trace analysis.   

\section{Future Work}
Distinguishing is considered when multiple clients connecting to the same server. Each connection is considered as an instance. In the server side all this instances have the same pipe name but different handler ID. However in the assembly trace level there is no way to match a client with it instance handler ID. In consequence, if the same content messages are being sent/received by different clients, when the user want to match the message pair between a client and the server, there is no way to distinguish the correct one from the assembly trace level. As a result, our tool will list all the matched content message event, regardless if it's from the interested client. The user can distinguish the correct ones for this client, if they have extra information.

