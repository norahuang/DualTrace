\newpage
\TOCadd{Abstract}

\noindent \textbf{Supervisory Committee}
\tpbreak
\panel

\begin{center}
\textbf{ABSTRACT}
\end{center}

Understanding the communication between programs can help the software security engineers understand the behaviour of a system and detect the vulnerabilities of a system. Assembly level execution traces are used for analyzing the communications between programs for the two reasons: 1) lack of source code of the running programs, 2) assembly execution traces provide more accurate run time behaviour information about the system. In this thesis, I present a communication analysis approach using assembly level execution traces. I first defined the communication model in the context of trace analysis. Then I developed a process and the necessary algorithms to identify the communications from a dual\_trace which consist of two assembly level execution traces. A prototype is developed for communication analysis. Finally, I conducted two experiments for communication analysis of interacting programs. These two experiments shows the usefulness of the designed communication analysis approach, the developed algorithms and the implemented prototype. 

