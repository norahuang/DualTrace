\newpage
\TOCadd{Abstract}

\noindent \textbf{Supervisory Committee}
\tpbreak
\panel

\begin{center}
\textbf{ABSTRACT}
\end{center}

Understanding the communications between programs can help software security engineers understand the behaviour of a system and detect vulnerabilities in a system. Assembly-level execution traces are used for this purpose for two reasons: 1) lack of source code of the running programs, and 2) assembly-level execution traces provide the most accurate run-time behaviour information. In this thesis, I present a communication analysis approach using such execution traces. I first model the message based communication in the context of trace analysis. Then I develop a method and the necessary algorithms to identify communications from a dual\_trace which consist of two assembly level execution traces. A prototype is developed for communication analysis. Finally, I conducted two experiments for communication analysis of interacting programs. These two experiments show the usefulness of the designed communication analysis approach, the developed algorithms and the implemented prototype. 

